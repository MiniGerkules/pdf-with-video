\documentclass{article}

\usepackage[utf8]{inputenc}
\usepackage[russian]{babel}
\usepackage{hyperref}

\title{Инструкция по написанию конспектов к видео лекциям}
\author{Посов Илья}
\date{\today}

\begin{document}
    \maketitle


    \section{Введение}
    Репозиторий \url{https://github.com/ipo-kio/pdf-with-video} содержит инструменты для написания конспектов к
    видео-лекциям.
    Внутри репозитория находятся:
    \begin{enumerate}
        \item \LaTeX-класс для оформления конспектов лекций;
        \item пакет \texttt{timestamps} для указания меток времени внутри конспектов;
        \item приложение, работающее в браузере, для проверки связи конспекта с видео;
        \item сами конспекты лекций.
    \end{enumerate}

    В дальнейших разделах предполагается, что вы сумели клонировать себе репозиторий.
    Инструкции по работе с git
    при необходимости будут написаны отдельно.


    \section{Установка}
    Чтобы писать конспекты в \LaTeX, необходимо настроить свой дистрибутив \TeX, чтобы он мог найти
    необходимые классы и пакеты.
    Напомню, что \LaTeX-документы начинаются с команды, указывающей класс документа, в нашем случае
    используется класс \texttt{lecture-notes}, и это указывается следующим образом:

    \begin{verbatim}\documentclass[russian]{lecture-notes}
    \end{verbatim}

    Простейший способ заставить \LaTeX увидеть этот класс~--- это просто скопировать файл lecture-notes.cls
    в папку с конспектом. Но этот метод нужно использовать только в крайнем случае, потому что класс будет
    постоянно изменяться, и будет очень неудобно постоянно копировать его новые версии. Правильный способ настроить
    класс~--- это добавить каталог с классом в список каталогов, которые ваш дистрибутив \TeX
    просматривет при поиске классов.

    В случае с MikTeX процесс описан по ссылке \url{https://miktex.org/kb/texmf-roots}, смотрите нижний раздел про
    добавление собственных texmf-директорий.
    Вам нужно добавить директорию \texttt{/src/texmf} из репозитория.

    В случае, если вы пользуетесь Linux с дистрибутивом texlive, создайте каталог \texttt{~/texmf/}, это каталог внутри
    домашнего каталога. И вставьте в нем символьную ссылку на каталог внутри репозитория:

    \begin{verbatim}
mkdir -p ~/texmf/tex/latex
ln -s репозиторий/src/texmf/tex/latex/video-timestamps \
      ~/texmf/tex/latex/video-timestamps
    \end{verbatim}


    \section{Написание конспекта}

    Шаблон конспекта расположен в файле \\ \texttt{src/texmf/doc/latex/video-timestamps/lecture-notes.tex}.

    Редакторов для \LaTeX много, я рекомендую \href{https://www.texstudio.org/}{TeXStudio}.

    Чтобы сделать свой конспект, скопируйте этот файл и сделайте в нем необходимые исправления. Сам файл
    содержит разметку, которая требуется для конспекта и примеры часто необходимых возможностей, например,
    как писать формулы и как вставлять изображения. В файле есть только примеры, более полная информация
    находится здесь, либо в учебниках \LaTeX. Я рекомендую учебники на сайте
    \href{https://www.overleaf.com/learn}{Overleaf} и \href{https://en.wikibooks.org/wiki/LaTeX}{LaTeX wiki book}.

    Готовые конспекты должны быть расположены в каталоге \texttt{scr/lectures}, по отдельному каталогу
    на каждую лекцию. Если курс содержит несколько лекций, значит, нужно создать каталог для курса, а внутри
    каталоги отдельных лекций.

    В репозиторий необходимо добавлять только tex файлы с текстом и изображения. Все остальные типы файлов, которые
    генерирует TeX, например, log или aux файлы, не должны попадать в репозиторий. Они явно игнорируются в .gitignore.


    \section{Изображения в конспекте}

    Примеры, как вставлять изображения, есть в шаблоне конспекта. Здесь обсудим выбор форматов изображений.

    Используйте формат jpg для фотографий. Скриншоты сохраняются в формате png.

    \subsection{Свои рисунки в конспекте}
    Если вы делаете свой рисунок, то есть три возможности:
    \begin{enumerate}
        \item растровое изображение в формате png;
        \item векторное изображение в формате svg;
        \item векторное изображение с помощью пакета \texttt{tikz}.
    \end{enumerate}

    Изображения в формате png лучше избегать. Их труднее редактировать, они теряют качество при масштабировании,
    и они занимают больше места, особенно, если у них несколько версий в git репозитории.

    Пакет \texttt{tikz}~--- это хороший вариант, но рисовать с его помощью сложно, этому надо учиться. Либо нужно
    найти программы для рисования, которые способны генерировать tikz код.

    Рекомендуется использовать формат svg. Для редактирования изображений в этом формате используйте программу
    Inkscape, это бесплатный многофункциональный редактор векторной графики типа Corel Draw или Adobe Illustrator.
    Его в любом случае потребуется установить, если вы решите пользоваться svg изображениями в \LaTeX-документах,
    потому что \LaTeX\ использует ее внутри себя для чтения svg.

    Чтобы svg изображения вставлялись в конспект, нужны дополнительные действия по настройке \LaTeX.
    Кроме установки Inkscape проследите, чтобы при компиляции конспекта был указан параметр компиляции
    \texttt{{-}-shell-escape}. При компиляции из командной строки пишите:
    \begin{verbatim}pdflatex --shell-escape my-notes.tex
    \end{verbatim}
    При компиляции в TeXStudio, отройте меню Options, внутри Configure TeXStudio, вкладка Commands, впишите
    в PDFLaTeX компилятор нужную опцию, получится что-то типа:
    \begin{verbatim}pdflatex -synctex=1 -interaction=nonstopmode --shell-escape %.tex
    \end{verbatim}
    Эта опция нужна, чтобы разрешить \LaTeX\ запускать внешние программы, в нашем случае Inkscape, в процессе
    компиляции. По-умолчанию это отключено из соображений безопасности.


    \section{Проверка конспекта в браузере с видео}
    Раздел пишется...

    \section{Важные правила при оформлении текста в LaTeX}
    Раздел в разработке...

\end{document}
