\documentclass[russian]{lecture-notes}

\usepackage[final]{graphicx}
\usepackage{subcaption}
\usepackage{timestamps}
\usepackage{hyperref}
\usepackage{float}
\usepackage{amsmath}
\usepackage{cancel}
\usepackage{amssymb}
\usepackage{tikz}
\usepackage{algorithm}
\usepackage{algpseudocode}

\cleartheorem{example*}
\theoremstyle{definition}
\newtheorem{example*}{Пример}[subsection]
\newcommand{\R}[2]{$#1\mathrel{R}#2$}
\newcommand{\UR}[2]{$#1\mathrel{\cancel{R}}#2$}
\newcommand{\divs}{\mathrel{\raisebox{-2pt}{\vdots}}}

\title{Лекция "Бинарные отношения"}
\lecturer{Поздняков Сергей Николаевич}
\notesauthor{Кацер Евгений}
\date{22 мая 2018 г.}
\youtubevideo{oXcW6PIXd4s}

\begin{document}
	\maketitle

\timestamp{00:39}

\section{Свойства бинарных отношений:}

\subsection{Рефлексивность}
\label{opr:refleks}
\begin{definition} 
	$R$~--- \underline{рефлексивно} на множестве $M$, если, по определению, $\forall x \in M$, на котором это отношение определено, \R{x}{x} ($x$ находится в отношении с самим собой).
\end{definition}

\begin{example}
	\begin{enumerate}
		\item Делимость~--- рефлексивное отношение, если рассматривается подмножество $\mathbb{Z}$, не содержащее 0. Например, на множестве $\mathbb{N}$.
		\item Строгое сравнение~--- нерефлексивное отношение (на любом множестве).
		\item Сравнение по модулю~--- рефлексивное отношение (на любом множестве).
	\end{enumerate}
\end{example}

Если рефлексивное отношение задано:
\begin{itemize}
	\item Матрицей, то на главной диагонали будут стоять 1.
	\begin{equation*}
	\left(
	\begin{array}{cccc}
	1 & a_{12} & \ldots & a_{1n}\\
	a_{21} & 1 & \ldots & a_{2n}\\
	\vdots & \vdots & \ddots & \vdots\\
	a_{n1} & a_{n2} & \ldots & 1
	\end{array}
	\right)
	\end{equation*}	
	
	\item Графом, то он имеет петли. Граф, в котором могут быть петли, называется \underline{псевдографом}.
	
	\begin{figure}[H]
		\centering
		\tikz{
			\path [thick, draw=black] (-1,-1) circle(5mm);
			\path [thick, draw=black] (1,-1) circle(5mm);
			\path [thick, draw=black] (0,0.5) circle(5mm);
			\draw [->, thick, rotate=45] (-1.915,0) arc(0:345:0.5 and 0.25);
			\draw [<-, thick, rotate=-45] (1.915,-0.1) arc(190:525:0.5 and 0.25);
			\draw [->, thick] (0,1) arc(-90:255:0.25 and 0.5);
		}
		\caption{Рефлексивность на графе}
		\label{fig:refleks}
	\end{figure}

\end{itemize}

\subsection{Антирефлексивность}
\label{opr:antirefleks}
\begin{definition}
	\timestamp{05:39}
	$R$~--- \underline{антирефлексивно} на множестве $M$, если, по определению, $\forall x \in M$, на котором это отношение определено, \UR{x}{x} ($x$ не находится в отношении с самим собой).
\end{definition}

\begin{example*}
	Взаимная простота~--- антирефлексивное отношение, например, на подмножестве $\mathbb{N}$, не содержащем 1.
\end{example*}

\begin{example*}
	Строго меньше~--- антирефлексивное отношение (на любом множестве).
\end{example*}

Если антирефлексивное отношение задано:
\begin{itemize}
	\item Матрицей, то на главной диагонали будут стоять 0.
	\begin{equation*}
	\left(
	\begin{array}{cccc}
	0 & a_{12} & \ldots & a_{1n}\\
	a_{21} & 0 & \ldots & a_{2n}\\
	\vdots & \vdots & \ddots & \vdots\\
	a_{n1} & a_{n2} & \ldots & 0
	\end{array}
	\right)
	\end{equation*}
	
	\item Графом, то у него не будет петель, будут только ребра.
	
	\begin{figure}[H]
		\centering
		\tikz{
			\path [thick, draw=black] (-1,-1) circle(5mm);
			\path [thick, draw=black] (1,-1) circle(5mm);
			\path [thick, draw=black] (0,0.5) circle(5mm);
			\draw [->, thick] (-0.75,-0.55) -- (-0.30, 0.10);
			\draw [<->, thick] (0.3, 0.1) -- (0.75,-0.55);
			\draw [<-, thick] (-0.5, -1) -- (0.5,-1);
		}
		\caption{\small Антирефлексивность на графе}
		\label{fig:antirefleks}
	\end{figure}
\end{itemize}

\begin{note}	
	Отношение может быть и не рефлексивным (\ref{opr:refleks}), и не антирефлексивным (\ref{opr:antirefleks}). Если такое отношение задается матрицей, то на её главной диагонали должны стоять и 0, и 1.
\end{note}

\subsection{Симметричность}
\label{opr:simmetr}
\begin{definition}
	\timestamp{08:50}
	$R$~--- \underline{симметрично} на множестве $M$, если, по определению, $\forall x, y \in M$, на котором это отношение определено, \R{x}{y} $\Rightarrow$ \R{y}{x} (если $x$ находится в отношении с $y$, то $y$ находится в отношении с $x$).
\end{definition}

\begin{example*}
	Взаимная простота~--- симметричное отношение, например, на подмножестве $\mathbb{N}$, не содержащем 1.
\end{example*}

\begin{example*}
	Равенство~--- симметричное отношение (на любом множестве).
\end{example*}

\begin{example*}
	Сравнение по модулю~--- симметричное отношение (на любом множестве).
\end{example*}

Если симметричное отношение задано:
\begin{itemize}
	\item Матрицей, то матрица симметричная.
	\begin{equation*}
	\left(
	\begin{array}{cccc}
	a_{11} & a_{12} & \ldots & a_{1n}\\
	a_{21} & a_{22} & \ldots & a_{2n}\\
	\vdots & \vdots & \ddots & \vdots\\
	a_{n1} & a_{n2} & \ldots & a_{nn}
	\end{array}
	\right) 
	= 
	\left(
	\begin{array}{cccc}
	a_{11} & a_{12} & \ldots & a_{1n}\\
	a_{21} & a_{22} & \ldots & a_{2n}\\
	\vdots & \vdots & \ddots & \vdots\\
	a_{n1} & a_{n2} & \ldots & a_{nn}
	\end{array}
	\right)^T
	\end{equation*}
	
	\item Графом, то вместе с каждым ребром будет ему обратное. В этом случае стрелку <<туда-обратно>> заменяют отрезком или дугой без указания направления. Если в графе все стрелки парные, его называют \underline{неориентированным}.
	\begin{figure}[H]
		\begin{subfigure}[H]{0.49\linewidth}
			\centering
			\tikz{
				\path [thick, draw=black] (-1,-1) circle(5mm);
				\path [thick, draw=black] (1,-1) circle(5mm);
				\path [thick, draw=black] (0,0.5) circle(5mm);
				\draw [<->,thick] (-0.75,-0.55) -- (-0.30, 0.10);
				\draw [<->,thick] (0.30, 0.10) -- (0.75,-0.55);
			}
			\subcaption{\small Первый вариант изображения}
			\label{fig:simmetr1}
		\end{subfigure}
		\hfill
		\begin{subfigure}[H]{0.49\linewidth}
			\centering
			\tikz{
				\path [thick, draw=black] (-1,-1) circle(5mm);
				\path [thick, draw=black] (1,-1) circle(5mm);
				\path [thick, draw=black] (0,0.5) circle(5mm);
				\draw [thick] (-0.75,-0.55) -- (-0.30, 0.10);
				\draw [thick] (0.30, 0.10) -- (0.75,-0.55);
			}
			\subcaption{\small Второй вариант изображения}
			\label{fig:simmetr2}
		\end{subfigure}
	\caption{\small Симметричность на графе}
	\label{fig:simmetr}
	\end{figure}
\end{itemize}

\subsection{Антисимметричность}
\label{opr:antisimmetr}
\begin{definition}
	\timestamp{11:44}
	$R$~--- \underline{антисимметрично} на множестве $M$, если, по определению, $\forall x, y \in M$, на котором это отношение определено, \R{x}{y} и \R{y}{x} $\Rightarrow x=y$ (если $x$ находится в отношении с $y$, и $y$ находится в отношении с $x$, то они совпадают).
\end{definition}

\begin{example*}
	Нестрогое сравнение~--- антисимметричное отношение (на любом множестве).
\end{example*}

\begin{example*}
	Равенство~--- антисимметричное отношение (на любом множестве).
\end{example*}

\begin{example*}
	Делимость~--- антисимметрично на множестве $\mathbb{N}$.
\end{example*}

Если антисимметричное отношение задано:
\begin{itemize}
	\item Матрицей, то если элемент матрицы = 1, то симметричный ему относительно главной диагонали~--- 0, если элемент = 0, то симметричный ему~--- любой и на главной диагонали могут стоять и 0, и 1. Например:
	\begin{equation*}
	\left(
	\begin{array}{cccc}
	0 & 0 & 1\\
	1 & 1 & 0\\
	0 & 0 & 1
	\end{array}
	\right) 
	\end{equation*}
	
	\item Графом, то у него не будет двойных стрелок <<туда и обратно>>.
	\begin{figure}[H]
		\centering
		\tikz{
			\path [thick, draw=black] (-1,-1) circle(5mm);
			\path [thick, draw=black] (1,-1) circle(5mm);
			\path [thick, draw=black] (0,0.5) circle(5mm);
			\draw [->, thick] (-0.75,-0.55) -- (-0.30, 0.10);
			\draw [<-, thick] (0.30, 0.10) -- (0.75,-0.55);
		}
		\caption{\small Антисимметричность на графе}
		\label{fig:antisimmetr}
	\end{figure}
\end{itemize}

\subsection{Асимметричность}
\label{opr:asimmetr}
\begin{definition}
	\timestamp{16:59}
	$R$~--- \underline{асимметрично} на множестве $M$, если, по определению, $\forall x, y \in M$, на котором это отношение определено, \R{x}{y} $\Rightarrow$ \UR{y}{x} (если $x$ находится в отношении с $y$, то $y$ не находится в отношении с $x$).
\end{definition}

\begin{example}
	Строгое сравнение~--- асимметричное отношение (на любом множестве).
\end{example}

Если асимметричное отношение задано:
\begin{itemize}
	\item Матрицей, то если элемент матрицы = 1, то симметричный ему относительно главной диагонали~--- 0, если элемент = 0, то симметричный ему~--- любой и на главной диагонали стоят 0. Например:
	\begin{equation*}
	\left(
	\begin{array}{cccc}
	0 & 0 & 1\\
	1 & 0 & 0\\
	0 & 0 & 0
	\end{array}
	\right) 
	\end{equation*}
	
	\item Графом, то его стрелки направлены всегда в одну сторону.
	\begin{figure}[H]
		\centering
		\tikz{
			\path [thick, draw=black] (-1,-1) circle(5mm);
			\path [thick, draw=black] (1,-1) circle(5mm);
			\path [thick, draw=black] (0,0.5) circle(5mm);
			\draw [->, thick] (-0.75,-0.55) -- (-0.30, 0.10);
			\draw [<-, thick] (0.30, 0.10) -- (0.75,-0.55);
		}
		\caption{\small Асимметричность на графе}
		\label{fig:asimmetr}
	\end{figure}
\end{itemize}

\begin{note*}
	Если отношение асимметрично, то оно и антирефлексивно, и антисимметрично.
\end{note*}

\begin{note*}
	Асимметричное отношение задает более жёсткие условия по сравнению с антисимметричным отношением.
\end{note*}

\subsection{Транзитивность}
\label{opr:tranzit}
\begin{definition}
	\timestamp{20:38}
	$R$~--- \underline{транзитивно} на множестве $M$, если, по определению, $\forall x, y, z \in M$, на котором это отношение определено, \R{x}{y} и \R{y}{z} $\Rightarrow$ \R{x}{z} (если $x$ находится в отношении с $y$, и $y$ находится в отношении с $z$, то $x$ находится в отношении с $z$).
\end{definition}

\begin{figure}[H]
	\begin{subfigure}[H]{0.49\linewidth}
		\centering
		\tikz{
			\path [thick, draw=black] (-1,-1) circle(5mm);
			\path [thick, draw=black] (1,-1) circle(5mm);
			\path [thick, draw=black] (0,0.5) circle(5mm);
			\draw [->, thick] (-0.75,-0.55) -- (-0.30, 0.10);
			\draw [->, thick] (0.30, 0.10) -- (0.75,-0.55);
			\draw [->, thick, dashed] (-0.5, -1) -- (0.5, -1);
			\coordinate [label=above:$y$] (y) at (0, 1);
			\coordinate [label=-135:$x$] (x) at (-1.3, -1.3);
			\coordinate [label=-45:$z$] (z) at (1.3, -1.3);
		}
		\subcaption{\small Отношение будет транзитивно, если добавить ребро из $x$ в $z$}
		\label{fig:tranzit(a)}
	\end{subfigure}
	\hfill
	\begin{subfigure}[H]{0.49\linewidth}
		\centering
		\tikz{
			\path [thick, draw=black] (-1,-1) circle(5mm);
			\path [thick, draw=black] (1,-1) circle(5mm);
			\path [thick, draw=black] (0,0.5) circle(5mm);
			\draw [->, thick] (-0.75,-0.55) -- (-0.30, 0.10);
			\draw [->, thick] (-0.5, -1) -- (0.5, -1);
			\coordinate [label=above:$y$] (y) at (0, 1);
			\coordinate [label=-135:$x$] (x) at (-1.3, -1.3);
			\coordinate [label=-45:$z$] (z) at (1.3, -1.3);
		}
		\subcaption{\small Отношение транзитивно}
		\label{fig:tranzit(b)}
	\end{subfigure}
	\caption{\small Транзитивность на графах}
	\label{fig:tranzit}
\end{figure}

\begin{note}
	Добавляя ребро из $x$ в $z$ мы осущесвляем транзитивное замыкание.
\end{note}

\begin{center}
	\timestamp{26:57}
	\section*{\LARGE\S 3 Отношение эквивалентности \label{par:otnosh}}
\end{center}

\begin{definition}
	\label{opr:otnosh}
	Отношение $R$ на множестве $M$ называется \underline{отношением } \underline{эквивалентности} по определению, если оно обладает следующими свойствами:
	\begin{enumerate}
		\item Рефлексивность (\ref{opr:refleks})
		\item Симметричность (\ref{opr:simmetr})
		\item Транзитивность (\ref{opr:tranzit})
	\end{enumerate}
\end{definition}

\begin{example*}
	Отношение подобия треугольников:
	\begin{enumerate}
		\item Рефлексивно, так как каждый треугольник подобен самому себе.
		\item Симметрично, так как если первый треугольник подобен второму, то и второй треугольник подобен первому.
		\item Транзитивно, так как если первый треугольник подобен второму, а второй подобен третьему, то первый треугольник подобен третьему.
	\end{enumerate}
\end{example*}
	
\begin{example*}
		Отношение сравнимости по модулю $m$ (\R{x}{y} $\Leftrightarrow x \equiv y \pmod m$ на множестве $M = \mathbb{Z}$):
		\begin{enumerate}
			\item $x\equiv x \pmod m$
			\item $x\equiv y \pmod m \Rightarrow y\equiv x \pmod m$
			\item $x\equiv y \pmod m, \ y\equiv z \pmod m \Rightarrow x\equiv z \pmod m$
		\end{enumerate}
\end{example*}

\begin{example*}
		Проверим то же самое по свойству сравнимости $x \equiv y \pmod m \Leftrightarrow (x-y) \divs m$, тогда:
		\begin{enumerate}
			\item $x-x=0 \Rightarrow (x-x) \divs m$
			\item $\begin{cases}
					(x-y) \divs m
					\\
					y-x=-(x-y)
					\end{cases}
					\Rightarrow (y-x) \divs m$
			\item Доказать, что $\begin{cases}
					(x-y) \divs m
					\\
					(y-z) \divs m
					\end{cases}
					\Rightarrow (x-z) \divs m$\\
					По свойству делимости, если $a \divs m$ и $b \divs m$, то и $(a+b) \divs m \Rightarrow \\ \Rightarrow \underbrace{((x-y)+(y-z))}_{(x-z)} \divs m \Rightarrow (x-z) \divs m$
		\end{enumerate}
\end{example*}


\begin{theorem}[о классах эквивалентности\label{teor:klass}]
	\timestamp{36:48}
	Если на множестве $M$ задано отношение эквивалентности $R$, то все множество $M$ разбивается на (непересекающиеся) классы, объединение которых есть $M$, так, что любые элементы из одного класса между собой эквивалентны (находятся в заданном отношении), а элементы из разных классов не эквивалентны.
\end{theorem}

\begin{figure}[H]
	\centering
	\tikz{
		\draw [thick] (2,-0.9) -- (3,1);
		\draw [thick] (3,1) -- (5,0.6);
		\draw [thick] (2,-0.9) -- (5,0.6);
		
		\draw [thick] (0,0) -- (0,2);
		\draw [thick] (0,0) -- (-4,1);
		\draw [thick] (0,0) -- (-2,1.4);
		\draw [thick] (-4,1) -- (-2,1.4);
		\draw [thick] (-4,1) -- (-3,3);
		\draw [thick] (-3,3) -- (-2,1.4);
		\draw [thick] (-3,3) -- (0,2);
		\draw [thick] (0,2) -- (-2,1.4);
		
		\draw[thick, black] (-3,3) arc (90:180:1.15 and 2);
		\draw[thick, black] (0,0) arc (-45:135:2 and 2.2);
		
		\path [thick, fill=white, draw=black] (0,0) circle(3mm);
		\path [thick, fill=white, draw=black] (-2,-0.9) circle(3mm);
		\path [thick, fill=white, draw=black] (2,-0.9) circle(3mm);
		\path [thick, fill=white, draw=black] (3,1) circle(3mm);
		\path [thick, fill=white, draw=black] (5,0.6) circle(3mm);
		\path [thick, fill=white, draw=black] (0,2) circle(3mm);
		\path [thick, fill=white, draw=black] (-3,3) circle(3mm);
		\path [thick, fill=white, draw=black] (-2,1.4) circle(3mm);
		\path [thick, fill=white, draw=black] (-4,1) circle(3mm);
		
		\coordinate [label=-135:$x$] (x) at (-4.15,0.85);
	}
	\caption{\small Классы эквивалентности, если отношение задано на графах}
	\label{fig:klass}
\end{figure}

\begin{note}
	На Рисунке \ref{fig:klass} граф неориентированный, петли в каждой вершине не нарисованы. Всего представлено 3 класса эквивалентности. Каждый из них является полным графом.
\end{note}

\begin{proof}
	Обозначим за $M_x=\{y \in M \mid yRx\}$. Для доказательства достаточно доказать, что любые 2 класса такого вида либо совпадают, либо не пересекаются.
	
	От противного: пусть нашлось два таких класса $M_x$ и $M_y$, что они не совпадают и пересекаются.
	
	\begin{figure}[H]
		\centering
		\tikz{
			\path [thick, draw=black] (-0.75,0) circle(1cm);
			\path [thick, draw=black] (0.75,0) circle(1cm);
			
			\path [thick, fill=black, draw=black] (-0.75,0) circle(1.3mm);
			\path [thick, fill=black, draw=black] (0.75,0) circle(1.3mm);
			\path [thick, fill=black, draw=black] (0,0) circle(1.3mm);
			
			\coordinate [label=below:$z$] (z) at (0,-0.1);
			\coordinate [label=-135:$x$] (x) at (-0.76,-0.1);
			\coordinate [label=-45:$y$] (y) at (0.76,-0.1);
		}
		\caption{\small Точки - множества подмножеств}
		\label{fig:dok1}
	\end{figure}
	
	Не умаляя общности, будем считать, что $x \notin M_y$ и $y \notin M_x$, а $\underbrace{z \in M_x}_{\text{\R{z}{x}}}$ и $\underbrace{z \in M_y}_{\text{\R{z}{y}}}$\newline
	
	\R{z}{x} $\Rightarrow$ \R{x}{z} по симметричности $\Rightarrow
	\begin{cases}
		\text{\R{x}{z}} 
		\\
		\text{\R{z}{y}}
	\end{cases} \Rightarrow \\ \Rightarrow [\text{по транзитивности}] \Rightarrow$ \R{x}{y} $\Rightarrow x \in M_y$~--- противоречие!!!
	
\end{proof}

\timestamp{50:14}
\begin{algorithm}[H]
	\caption{Алгоритм построения классов эквивалентности на основе построения компонент связности}
	\label{alg:klass}
	\begin{algorithmic}[1]
		\State Инициализация:
		\For {$i$ от $1$ до $n$}
		\State $color(i) := 0$
		\EndFor
		\State $C := 0$
		\Comment{номер цвета}
		\Statex
		\For {$i$ от $1$ до $n$}
		\If{$color(i) = 0$}
		\State $C := C + 1$
		\State \Call{paint}{$i, n, C$}
		\Comment{Раскрасить в цвет $C$ все элементы, находящиеся в отношении $R$ с $i$ номером и имеющие больший, либо равный номер.}
		\EndIf
		\EndFor
		\Statex
		\Procedure{paint}{$i, n, C$}
		\For {$j$ от $i$ до $n$}
		\If {$jRi$}
		\State $color(j) := C$
		\EndIf
		\EndFor
		\EndProcedure
	\end{algorithmic}
\end{algorithm}

\begin{figure}[H]
	\centering
	\tikz{
		\draw[thick, black] (1.5,0.3) arc (0:180:2.5 and 1.7);
		\draw[thick, black] (-0.5,0.3) arc (0:180:1 and 0.7);
		\draw[thick, black] (-0.5,0.3) arc (0:180:0.5 and 0.35);
		\draw[thick, black] (-1.5,0.3) arc (0:180:0.5 and 0.35);
		\draw[thick, black] (2.5,-0.3) arc (0:-180:1 and 0.7);
		
		\path [thick, fill=blue, draw=black] (0.5,0) circle(3mm);
		\path [thick, fill=red, draw=black] (-0.5,0) circle(3mm);
		\path [thick, fill=black, draw=black] (1.5,0) circle(3mm);
		\path [thick, fill=red, draw=black] (-1.5,0) circle(3mm);
		\path [thick, fill=blue, draw=black] (2.5,0) circle(3mm);
		\path [thick, fill=red, draw=black] (-2.5,0) circle(3mm);
		\path [thick, fill=white, draw=black] (3.5,0) circle(3mm);
		\path [thick, fill=black, draw=black] (-3.5,0) circle(3mm);
		
		\coordinate [label=135:$1$] (1) at(-3.65,0.15);
		\coordinate [label=45:$1$] (1) at(1.65,0.15);
		\coordinate [label=below:$2$] (2) at(-2.5,-0.3);
		\coordinate [label=below:$2$] (2) at(-1.5,-0.3);
		\coordinate [label=below:$2$] (2) at(-0.5,-0.3);
		\coordinate [label=above:$3$] (3) at(0.5,0.3);
		\coordinate [label=above:$3$] (3) at(2.5,0.3);
		\coordinate [label=below:$4$] (4) at(3.5,-0.3);
	}
	\caption{\small Пример работы алгоритма}
	\label{fig:examplealg}
\end{figure}

\begin{center}
	\timestamp{58:27}
	\section*{\LARGE\S 4 Отношение порядка}
\end{center}

\begin{definition}
	$R$~--- \underline{отношение (нестрогого) порядка}, если, по определению, $R$ обладает свойствами:
	\begin{enumerate}
		\item Рефлексивность (\ref{opr:refleks})
		\item Антисимметричность (\ref{opr:antisimmetr})
		\item Транзитивность (\ref{opr:tranzit})
	\end{enumerate}
\end{definition}

\begin{remark}
	$R$~--- \underline{отношение строгого порядка}, если оно обладает свойствами:
	\begin{enumerate}
		\item Антирефлексивность (\ref{opr:antirefleks})~--- можно не писать, так как если отношение асимметрично, то оно и антирефлексивно
		\item Асимметричность (\ref{opr:asimmetr})
		\item Транзитивность (\ref{opr:tranzit})
	\end{enumerate}
\end{remark}

\begin{example*}
	\label{examp:1}
	\R{x}{y} $\Leftrightarrow x \divs y$ на множестве $\mathbb{N}$. Проверим свойства:
	\begin{enumerate}
		\item $x \divs x \ \ \ \forall x \in \mathbb{N}$
		\item $x \divs y \text{ и } y \divs x \Rightarrow x=y$
		\item $x \divs y \text{ и } y \divs z \Rightarrow x \divs z$
	\end{enumerate}
\end{example*}

\begin{example*}
	\label{examp:2}
	\R{x}{y} $\Leftrightarrow x \leqslant y$ на множестве $\mathbb{R}$. Проверим свойства:
	\begin{enumerate}
		\item $x \leqslant x$
		\item $x \leqslant y \text{ и } y \leqslant x \Rightarrow x=y$
		\item $x \leqslant y \text{ и } y \leqslant z \Rightarrow x\leqslant z$
	\end{enumerate}
\end{example*}

\begin{remark}
	\timestamp{65:02}
	Существует свойство 4, которым обладает пример~\ref{examp:2} и не обладает пример~\ref{examp:1}.\\
	{\bf Свойство 4:}
	$\forall x,y \in M$ верно \R{x}{y} или \R{y}{x}.
\end{remark}

\begin{definition}
	Отношение порядка, удовлетворяющее свойству 4, называется \underline{отношением линейного порядка}, иначе (если свойство 4 не выполняется для какой-либо пары элементов) \underline{отношением частичного порядка}.
\end{definition}

\begin{example}
\end{example}

\begin{figure}[H]
	\centering
	\tikz{
		\draw [thick] (-0.8,0.5) -- (0.8,0.5) -- (0.8,1.2) -- (-0.8,1.2) -- cycle;
		\draw [thick] (-1.3,-1) -- (1.3,-1) -- (1.3,-0.3) -- (-1.3,-0.3) -- cycle;
		\draw [thick] (-4.3,-1) -- (-1.7,-1) -- (-1.7,-0.3) -- (-4.3,-0.3) -- cycle;
		\draw [thick] (1.9,-1) -- (4.1,-1) -- (4.1,-0.3) -- (1.9,-0.3) -- cycle;
		\draw [thick] (-6.7,-2.5) -- (-3.4,-2.5) -- (-3.4,-1.8) -- (-6.7,-1.8) -- cycle;
		\draw [thick] (-1,-2.5) -- (1,-2.5) -- (1,-1.8) -- (-1,-1.8) -- cycle;
		
		\draw [->, thick] (0,0.5) -- (0,-0.3);
		\draw [->, thick] (0,0.5) -- (3,-0.3);
		\draw [->, thick] (0,0.5) -- (-3,-0.3);
		\draw [->, thick] (-3,-1) -- (-5.05,-1.8);
		\draw [->, thick] (0,-1) -- (0,-1.8);
		\draw [->, thick] (-0.8,0.85) arc (90:180:4.5 and 2.65);
		\draw [->, thick] (0.8,0.85) arc (90:0:4.5 and 3) -- (1,-2.15);
		
		\coordinate [label=above:Ректор] (A) at (0,0.5);
		\coordinate [label=above:Проректор 1] (B1) at (-3,-1);
		\coordinate [label=above:Проректор 2] (B2) at (0,-1);
		\coordinate [label=above:Секретарь] (B3) at (3,-1);
		\coordinate [label=above:Зав. кафедрой 1] (C1) at (-5,-2.5);
		\coordinate [label=above:Завхоз 1] (C2) at (0,-2.42);
	}
	\caption{\small Отношение частичного порядка}
	\label{fig:otnchastpor}
\end{figure}

\timestamp{70:28}
Идея \underline{топологической сортировки} - добавить к частичному порядку новые связи, чтобы он стал линейным. Далее мы уточним это понятие.

\begin{example}
	Совершим топологическую сортировку в предыдущем примере (\ref{fig:otnchastpor}).
\end{example}

\begin{figure}[H]
	\centering
	\framebox{\begin{tikzpicture}[scale=0.8, every node/.style={transform shape}]
		
		\draw [thick] (-8,0) -- (-6.4,0) -- (-6.4,0.7) -- (-8,0.7) -- cycle;
		\draw [thick] (-6,0) -- (-3.4,0) -- (-3.4,0.7) -- (-6,0.7) -- cycle;
		\draw [thick] (-3,0) -- (-0.4,0) -- (-0.4,0.7) -- (-3,0.7) -- cycle;
		\draw [thick] (0.05,0) -- (3.3,0) -- (3.3,0.7) -- (0.05,0.7) -- cycle;
		\draw [thick] (3.8,0) -- (5.55,0) -- (5.55,0.7) -- (3.8,0.7) -- cycle;
		\draw [thick] (6.05,0) -- (8.15,0) -- (8.15,0.7) -- (6.05,0.7) -- cycle;
		
		\draw [->, thick] (-7.2,0.7) arc (180:0:1.25 and 0.5);
		\draw [->, thick] (-7.2,0.7) arc (180:0:2.7 and 0.8);
		\draw [->, thick] (-7.2,0.7) arc (180:0:4.5 and 1.1);
		\draw [->, thick] (-7.2,0.7) arc (180:0:6 and 1.4);
		\draw [->, thick] (-7.2,0.7) arc (180:0:7.2 and 1.7);
		
		\draw [->, thick] (-4.7,0) arc (-180:0:1.5 and 0.5);
		\draw [->, thick] (-4.7,0) arc (-180:0:3.15 and 0.8);
		\draw [->, thick] (-4.7,0) arc (-180:0:4.65 and 1.1);
		\draw [->, thick] (-4.7,0) arc (-180:0:5.85 and 1.4);
		
		\draw [->, thick] (-1.6,0.7) arc (180:0:1.6 and 0.5);
		\draw [->, thick] (-1.6,0.7) arc (180:0:3.1 and 0.8);
		\draw [->, thick] (-1.6,0.7) arc (180:0:4.3 and 1.1);
		
		\draw [->, thick] (1.815,0) arc (-180:0:1.5 and 0.5);
		\draw [->, thick] (1.815,0) arc (-180:0:2.7 and 0.8);
		
		\draw [->, thick] (5.55,0.35) -- (6.05,0.35);
		
		\coordinate [label=above:Ректор] (A) at (-7.2,0);
		\coordinate [label=above:Проректор 1] (B1) at (-4.7,0);
		\coordinate [label=above:Проректор 2] (B2) at (-1.7,0);
		\coordinate [label=above:Зав. кафедрой 1] (C1) at (1.7,0);
		\coordinate [label=above:Завхоз 1] (C2) at (4.7,0.08);
		\coordinate [label=above:Секретарь] (B3) at (7.1,0);
	\end{tikzpicture}}
	\caption{\small Топологическая сортировка}
	\label{fig:topsort}
\end{figure}

\begin{definition}
	\timestamp{73:16}
	$L$~--- \underline{согласовано} с $R$, если, $\forall x,y \in M$: \R{x}{y} $\Rightarrow x \mathrel{L} y$.
\end{definition}

\noindent{\bf Постановка задачи топологической сортировки. }Построить отношение линейного порядка $L$, согласованное с заданным отношением частичного порядка $R$.

\timestamp{76:06}
\begin{algorithm}[H]
	\caption{Алгоритм топологической сортировки}\label{alg:topsort}
	\begin{algorithmic}[1]
		\State $M$~--- конечное множество
		\State $L:=R$
		\Statex
		\While {$M \ne 0$}
		\State Найти минимальный эл-т $m$ множества $M$ относительно $R$.
		\State Добавим к $L$ пары $(m,x)$ такие, что $x \in M$.
		\State Удалим $m$ из $M$.
		\EndWhile
	\end{algorithmic}
\end{algorithm}

Обоснуем алгоритм:

\begin{definition}
	\label{opr:min}
	$x$~--- \underline{минимальный элемент} в $M$ относительно $R$, если не существует $m \in M$: \R{m}{x}.
\end{definition}

\begin{lemma}
	\label{lem:min}
	Если $M$~--- конечно, то минимальный элемент существует.
\end{lemma}

\begin{proof}
	\label{proof:min}
	От противного: пусть минимальный элемент не существует. Возьмем какой-нибудь элемент $x_1$. Для него найдется элемент меньше ($x_2$). Для $x_2$ найдется ещё меньший элемент~--- $x_3$, но, так как у нас конечное число элементов, где-то должен образоваться цикл. Пусть перед $x_i$ стоит элемент $x_j$, то есть существует условие $x_jRx_i$ (между повторением $x_i$ все элементы разные).
	\begin{center}
		$\underbrace{x_i \dots x_jRx_i}_{\text{т.к. число эл-тов конечно}} \dots x_3Rx_2Rx_1$
	\end{center}
	Проверим кусок элементов от $x_i$ до $x_j$ включительно:
	\begin{center}
		$x_iRx_k \dots Rx_j \Rightarrow \text{[по транзитивности]} \Rightarrow x_iRx_j$
	\end{center}
	$\begin{cases}
		x_iRx_j
		\\
		x_jRx_i
	\end{cases} \Rightarrow \text{[в силу симметрии]} \Rightarrow x_i=x_j$~--- противоречие!!!
	
\end{proof}

\begin{example}
\end{example}

\begin{figure}[H]
	\centering
	\tikz{
		\draw [thick] (-0.5,0) -- (0.5,0) -- (0.5,0.5) -- (-0.5,0.5) -- cycle;
		\draw [thick] (-1.5,-1) -- (-0.5,-1) -- (-0.5,-0.5) -- (-1.5,-0.5) -- cycle;
		\draw [thick] (0.5,-1) -- (1.5,-1) -- (1.5,-0.5) -- (0.5,-0.5) -- cycle;
		\draw [thick] (-0.25,-2) -- (0.75,-2) -- (0.75,-1.5) -- (-0.25,-1.5) -- cycle;
		\draw [thick] (1.25,-2) -- (2.25,-2) -- (2.25,-1.5) -- (1.25,-1.5) -- cycle;
		
		\draw [->, thick] (0,0) -- (-1,-0.5);
		\draw [->, thick] (0,0) -- (1,-0.5);
		\draw [->, thick] (1,-1) -- (0.25,-1.5);
		\draw [->, thick] (1,-1) -- (1.75,-1.5);
	}
	\caption{\small У этого множества 3 минимальных элемента}
\end{figure}

\begin{proof}
	Докажем, что алгоритм строит согласованное отношение линейного порядка.
	\begin{enumerate}
		\item Отношение согласованно, так как в процессе алгоритма мы только добавляем (все отношения. которые были, останутся).
		\item Отношение линейного порядка. Сопоставим каждому элементу $m \in M$ номер, которым мы его удалили из $M$.
		
		\begin{figure}[H]
			\centering
			\tikz{
				\path [thick, draw=black] (-3,0) circle(2mm);
				\path [thick, draw=black] (-2,0) circle(2mm);
				\path [thick, fill=black] (-1.15,0) circle(0.5mm);
				\path [thick, fill=black] (-1,0) circle(0.5mm);
				\path [thick, fill=black] (-0.85,0) circle(0.5mm);
				\path [thick, draw=black] (0,0) circle(2mm);
				\path [thick, fill=black] (1.15,0) circle(0.5mm);
				\path [thick, fill=black] (1,0) circle(0.5mm);
				\path [thick, fill=black] (0.85,0) circle(0.5mm);
				\path [thick, draw=black] (2,0) circle(2mm);
				\path [thick, fill=black] (3.15,0) circle(0.5mm);
				\path [thick, fill=black] (3,0) circle(0.5mm);
				\path [thick, fill=black] (2.85,0) circle(0.5mm);
				\path [thick, draw=black] (4,0) circle(2mm);
				
				\draw [->, thick] (-3,0.2) arc (180:0:0.5 and 0.25);
				\draw [->, thick] (-3,0.2) arc (180:0:1.5 and 0.55);
				\draw [->, thick] (-3,0.2) arc (180:0:2.5 and 0.85);
				\draw [->, thick] (-3,0.2) arc (180:0:3.5 and 1.15);
				\draw [->, thick] (-2,-0.2) arc (-180:0:1 and 0.25);
				\draw [->, thick] (-2,-0.2) arc (-180:0:2 and 0.55);
				\draw [->, thick] (-2,-0.2) arc (-180:0:3 and 0.85);
				
				\coordinate [label=below:1] (1) at(-3,-0.2);
				\coordinate [label=-135:2] (2) at(-1.9,-0.2);
				\coordinate [label=-45:i] (i) at(0,-0.2);
				\coordinate [label=-45:j] (j) at(2,-0.2);
				\coordinate [label=-45:n] (n) at(4,-0.2);
			}
			\caption{\small Сопоставим элементам номера}
			\label{fig:sopostav}
		\end{figure}
		
		Полученное отношение совпадает с отношением  $\leqslant$ на множестве номеров. А про отношение $\leqslant$ мы знаем, что это отношение линейного порядка.
	\end{enumerate}

\end{proof}

\end{document}
